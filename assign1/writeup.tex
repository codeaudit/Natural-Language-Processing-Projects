

Tradeoffs

Most counts in the trigram counter are very small and waste space as they are allotted 32 bits. In the training data, only around 150 trigrams appeared more than $2^16$ times. Because of this, I experimented with using 16-bit shorts to store trigram counts, and the results showed a 100MB reduction in memory usage, and a drop in BLEU from 24.8 to 24.7, which is a small amount to give up for the amount of memory saved. A small implementation was then tested to check for counter overflows and stop the counter at $2^16$ instead of overflowing, which would bring the counts closer to the actual count that the short was not capable of storing. However, this proved insignificant both in terms of additional decoding time to perform such check as well as BLEU, most likely due to some trigram counts reaching as high as around $2^19$, so using $2^16$ instead of a random short made little difference in the calculations.

The same modification was then made to the bigram counter. The bigram counter had significantly more counts that overflowed, while still being a tiny percentage of all bigrams. The two bigram fertility counters each had less than 50 overflows, however. Running the test with these modifications, there was another 100MB drop in memory use. The reduction per counter from these three counters switching to 16 bit from 32 bit was smaller as the counters were smaller due to there being fewer bigrams than trigrams. The BLEU score dropped from 24.7 to 23.5, which is quite a tradeoff for the memory reduction.


